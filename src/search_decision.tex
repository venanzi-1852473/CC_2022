\section{Search vs Decision}\label{sec:search_decision}

Consider $\varphi$ a boolean formula.

\begin{equation}\label{eq:decision}
    Decision(\varphi) := 
    \begin{cases}
        SAT \\
        \lnot SAT
    \end{cases}
\end{equation}

\begin{equation}\label{eq:search}
    Search(\varphi) := 
    \begin{cases}
        \alpha  & \text{a SAT assignment of } \varphi\\
        \perp   & \text{if } \varphi \text{ is UNSAT}
    \end{cases}
\end{equation}

Solving Search we can easily also solve Decision. Can we go in the opposite direction?
We can, but it requires a bit more thought.

Let $\varphi$ have $n$ variables. Starting from $\varphi$ with no variables fixed, we branch out all possible assignments.
Each time we branch we fix the value of a variable and we ask the Decision procedure if the formula, with the fixed values, is satisfiable.
In this way we also solve the Search problem.