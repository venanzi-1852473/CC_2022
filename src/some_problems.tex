\section{Some problems}\label{sec:some_problems}

\subsection{SAT and variations of it}\label{subsec:problem_sat}

    Let $\varphi$ be a boolean formula. The following problems can be defined for any kind of boolean formula, albeit it is often the case that we consider $\varphi$ to be in CNF form.

    We often denote an assignment of the variables of a formula with the letter $\alpha$.

    \begin{definition}[SAT]\label{def:sat}
        \[ SAT = \{ \varphi \st \text{there is an assignment } \alpha \text{ s.t. } \varphi(\alpha) = 1 \} \]    
    \end{definition}

    \begin{definition}[$k$-SAT]\label{def:k_sat}
        Fix $k$, we consider $\varphi$ to be a CNF boolean formula s.t. each clause contains exactly $k$ literals.
        \[ k\text{-}SAT = \{ \varphi \st \text{there is an assignment } \alpha \text{ s.t. } \varphi(\alpha) = 1 \} \]  
    \end{definition}

    


\subsection{Problems on graphs}\label{subsec:problem_graphs}

    For the following subsection the letter $G$ will denote a graph.

    \subsubsection{Independent Set}\label{subsubsec:problem_indset}
        \begin{definition}[Independent Set]\label{def:ind_set}
            $S \subseteq V(G)$ is an independent set if $\forall x,y \in S$ $\{ x,y \} \not\in E(G)$.
        \end{definition}

        So we define the decision problem of finding if a graph has an independent set as follows:
        \[ IndSet = \{ (G,k) \st G  \text{ has an independent set of size } \geq k \} \]

        \begin{theorem}\label{thm:indset_npc}
            3-SAT $\preduction$ IndSet
        \end{theorem}
        
        Also, it is easy to see that IndSet is in \textbf{NP}, so it is \textbf{NP}-Complete.


    \subsubsection{Clique}\label{subsubsec:problem_clique}
        Let $k$ be an integer.
        
        \begin{definition}[Clique]\label{def:clique}
            \[ Clique = \{ (G,k) \st G \text{ has a clique of size } k \} \]
        \end{definition}

        Observe that $(G,k) \in Clique \leftrightarrow (\bar{G},k) \in IndSet$. So $Clique$ is \textbf{NP}-Complete.


    \subsubsection{Vertex Cover}\label{subsubsec:problem_vertexcover}
        \begin{definition}[Vertex cover]\label{def:vertcover}
            $C \subseteq V(G)$ is a vertex cover if $\forall \{ x,y \} \in E(G)$ $x$ or $y$ are in $C$, and $C$ covers all the edges; i.e. each edge of the graph has at least one node in the cover.
        \end{definition}

        So we define the decision problems as follows:
        \[ VC = \{ (G,t) \st G \text{ has a vertex cover of size }  \leq t \} \]

        It is known that the complement of an \textit{independent set} is always a \textit{vertex cover}.
        So we can prove that $IndSet \preduction VC$ via the following mapping: $(G,k) \mapsto (G, \cardinality{V(G)} - k)$.
        This proves that $VC$ is \textbf{NP}-Complete.

    
    \subsubsection{Eulerian graph}\label{subsubsec:problem_euleriangraph}
        \begin{definition}[Eulerian walk]\label{def:eulerian_walk}
            A walk on a graph is said to be eulerian is it touches all edges exactly once.
        \end{definition}
        
        A graph is said to be \textit{eulerian} if it contains an eulerian walk.

        This problem is actually solvable in polynomial time; i.e. is in \textbf{P}.

        To check if a graph is eulerian we can do as follows.
        For each vertex calculate $d = indegree - outdegree$; if one of the two following conditions hold then the graph is eulerian:
        \begin{itemize}
            \item $d=0$ for each vertex
            \item one vertex has $d=1$, one vertex has $d=-1$ and all other vertices have $d=0$
        \end{itemize}

    \subsubsection{Hamiltonian path}\label{subsubsec:problem_hamiltonianpath}
        \begin{definition}[Hamiltonian path]
            A walk on a graph is said to be Hamiltonian if it touches all nodes exactly once.
        \end{definition}

        The statement is similar to the one for Eulerian walks, but this problem is actually in \textbf{NP}.