\section{coNP}\label{sec:conp}

\subsection{Definition}\label{subsec:conp_def}
    By definition, \textbf{coNP} is the following set:
    \[ coNP = \{ L \st \bar{L} \in NP \} \]

    This definition can be deceiving, beacause it might look like that $coNP = \bar{NP}$, but we know that this is not true (i.e. $coNP \neq \bar{NP}$).
    So we give an alternative definition.

    \begin{definition}[\textbf{coNP}]\label{def:coNP}
        A language $L$ is in \textbf{coNP} if there is a TM $M(\cdot, \cdot)$ s.t.
        \begin{enumerate}
            \item $M(x,y)$ runs in polytime \wrt{} $\cardinality{x}$ ($x$ is the input and $y$ the witness)
            \item $x \in L$, then for all $y$ $M(x,y) = 1$
            \item $x \not\in L$, there exists $y$ s.t. $M(x,y) = 0$
        \end{enumerate}
    \end{definition}

    Recall the definition of~\nameref{def:np}; the consequences of a string being in a language are swapped for \textbf{NP} and \textbf{coNP}.
    That is, if a problem is in \textbf{NP} we need a single witness to verify that an instance is in the problem, while in \textbf{coNP} we need that all witnesses verify that an instance is in the problem.
    The opposite goes for an instance being not in the language.


\subsection{Some problems}\label{subsec:conp_problems}
    Some \textbf{NP} problems:
    \begin{itemize}
        \item SAT
        \item FALSIFIABLE (set of boolean formulas that can be falsified)
    \end{itemize}

    Some \textbf{coNP} problems (negations of the above \textbf{NP} problems):
    \begin{itemize}
        \item UNSAT
        \item TAUT (tautologies)
    \end{itemize}

    \begin{claim}
        TAUT is \textbf{coNP}-Complete.
    \end{claim}

    \begin{proof}
        Let $L \in coNP$. We want to prove that $L \preduction TAUT$.

        Surely $\bar{L} \in NP$, so $\bar{L} \preduction SAT$, via some reduction $f$ (that outputs a CNF).
        \[ x \in \bar{L} \longleftrightarrow f(x) \in SAT \]
        \[ x \in L \longleftrightarrow f(x) \not\in SAT \]
        \[ x \in L \longleftrightarrow f(x) \in UNSAT \]
        \[ x \in L \longleftrightarrow \lnot f(x) \in TAUT \]
    \end{proof}