\section{Introduction to complexity classes}\label{sec:intro_compl_classes}


\subsection{Uncomputable functions}\label{subsec:uncomp_funcs}
    There are functions that can't be computed. They are also said to be \textit{undecidable}.

    \subsubsection{$UC$}\label{subsubsec:uc}
        Consider the following function:
        \begin{equation}\label{eq:uc}
            UC(\alpha) = 
            \begin{cases}
                0 & \text{if } M_\alpha(\alpha) = 1 \\
                1 & \text{otherwise}
            \end{cases}
        \end{equation}

        Through \textit{diagonalization} we can prove that $UC$ is uncomputable.

    \subsubsection{$HALT$}\label{subsubsec:halt}
        \begin{equation}\label{eq:uc}
            HALT(\alpha, x) = 
            \begin{cases}
                1 & \text{if } M_\alpha(x) \text{ terminates in finite steps} \\
                0 & \text{otherwise}
            \end{cases}
        \end{equation}

        We can prove that $HALT$ is uncomputable by \textit{reduction}: if $HALT$ is computable, then we can compute $UC$.
        For the definition of \textit{polynomial reductions} see \nameref{def:karp_reduction}.


\subsection{Time classes}\label{sec:time_classes}

    \begin{definition}[\dtime]\label{def:dtime}
        Let $f : \B^* \rightarrow \B$ be a function and $T : \mathbb{N} \rightarrow \mathbb{N}$ be a time function.
        Consider the set $L_f = \{ x \st f(x) = 1 \}$.

        $L \in \dtime[T]$ if there is a TM $M$ s.t.
        \begin{itemize}
            \item $M$ runs in time $O(T(n))$
            \item $M$ decides $L$
        \end{itemize}
    \end{definition}

    \subsubsection{\PP}\label{subsubsec:p}
        \begin{definition}[\PP]\label{def:p}
            \[ P = \bigcup_{c > 0} \dtime[n^c] \]

            We will consider any set in \PP{} to be efficiently computable.
        \end{definition}

    \subsubsection{\NP}\label{subsubsec:np}
        Problems in \NP{} capture the idea of puzzles.
        Given a puzzle it is hard to solve it, but given a possible solution to the puzzle, it is easy to check if such solution is correct.

        \begin{definition}[\NP]\label{def:np}
            Let $L$ be a language. $L \in NP$ if there is a machine $V(\cdot, \cdot)$ and a polynomial $p$ s.t.
            \begin{enumerate}
                \item $\forall x, y$ $V(x,y)$ runs in time $p(\cardinality{x})$ (\textbf{efficienct})
                \item $x \in L$, $\exists y$ s.t. $\cardinality{y} \leq p(\cardinality{x})$ and $V(x,y) = 1$ (\textbf{completeness})
                \item $x \not\in L$, $\forall y$ $V(x,y) = 0$ (\textbf{soundness})
            \end{enumerate}
        \end{definition}

        $x$ is an input for an \NP{} problem, and $y$ is a \textbf{witness}\label{witness}, a possible solution to the problem.
        $V$ is a \textit{verifier machine}, that verifies if $y$ is a \textit{valid witness} for $x$; $V$ runs in polynomial time \wrt $\cardinality{x}$ and can't be fooled (i.e. soundness).

    We know that $P \subseteq NP$: given a \PP{} problem we can just give it a witness, ignore the witness and solve the problem directly in polynomial time.

    \subsubsection{\EXP}\label{subsubsec:exp}
        \begin{definition}[\EXP]\label{def:exp}
            \[ EXP = \bigcup_{c > 0} \dtime[2^{n^c}] \]
        \end{definition}

        It is clear that
        \[ P \subseteq NP \subseteq EXP \]
        We also know that $P \subsetneqq EXP$, which means that there are problems strictly not solvable in polytime, and require exponential time.

    
\subsection{Reductions}\label{sec:reductions}

    Problem $A$ \textbf{reduces} to problem $B$ when: given an efficient algorithm for $B$, we can use it to build an efficient algorithm to solve $A$.
    
    We say that $A$ is "easier" than $B$.

    \begin{definition}[Karp reduction]\label{def:karp_reduction}
        Such reductions are often called just \textit{reduction} or \textit{polytime (computable) reductions}.

        A language $L \subseteq \B^*$ is \textit{polytime (Karp) reducible} to a language $L' \subseteq \B^*$ if there exists a polytime computable function $\starbtob$ s.t.
        \[ x \in L \Longleftrightarrow f(x) \in L' \]

        So, if $L$ is reducible to $L'$ we write $L \preduction L'$.
    \end{definition}


\subsection{NP-Hardness}\label{subsec:np_hardness}
    A language $S$ is NP-Hard if $\forall L \in$\NP $L \preduction S$.
    So an NP-Hard problem is harder than any other problem in \NP.

\subsection{NP-Completeness}\label{subsec:np_completeness}
    A problem $L$ is \textbf{NP-Complete} if:
    \begin{itemize}
        \item $L$ is \NP-Hard
        \item $L \in$ \NP
    \end{itemize}

    Some properties:
    \begin{itemize}
        \item if $A \preduction B$ and $B \preduction C$, then $A \preduction C$
        \item if $L$ is \NP-Hard and $L \in P$, then \PP $=$ \NP
        \item if $L$ is \NP-Complete then $L \in \PP$ iff \PP $=$ \NP
    \end{itemize}

    \begin{theorem}[Cook-Levin]\label{thm:cook_levin}
        The following are \NP-Complete problems:
        \begin{enumerate}
            \item SAT
            \item 3-SAT
            \item Circuit-SAT
        \end{enumerate}
    \end{theorem}


\subsection{On \NP}\label{subsec:on_np}
    \NP{} stands for \textbf{Nondeterministic Polynomial}, because a problem in \NP{} follows the computation of a \textbf{NonDeterministic Tugin Machine} (\textbf{NDTM}).

    In a normal TM the computation follows a chain, it is, indeed, deterministic. IN a NDTM computation branches out, in a binary tree fashion; e.g. given $n$ boolean variables, a NDTM, for each variable, tries all possible computations with such variable both set as $0$ as well as 1, so at the end it tried all possible $2^n$ combinations.

    A NDTM $M$ runs in time $T(n)$ if all paths of the computation of $M(x)$, for any input $x$, terminate within time $T(\cardinality{x})$.

    If a language $L$ is decided by a NDTM $M$ in time $T(n)$, then:
    \begin{itemize}
        \item $M$ runs in time $T(n)$
        \item $x \in L$, then there is a path s.t. $M(x) = 1$ (i.e. it accepts)
        \item $x \not\in L$, then for all paths $M(x) = 0$ (i.e. always rejects)
    \end{itemize}


    \begin{definition}[\ntime]\label{def:ntime}
        \[ \ntime[T(n)] = \{ L \st L \text{ is decided by NDTM in time } T(n) \} \]
    \end{definition}

    So we can give the following alternative definition for \NP:
    \[ \NP = \bigcup_{c > 0} \ntime[n^c] \]

%\subsection{\skipped}
%\begin{center}
%    \fbox{I'm skipping, from the notebook, \textit{Boolean Circuits (Anticipation)}}
%\end{center}